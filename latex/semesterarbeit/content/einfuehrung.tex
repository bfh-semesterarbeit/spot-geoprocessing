\section{Einführung}
Ich arbeite als GIS-Spezialist bei der swisstopo\footnote{\href{https://www.swisstopo.ch}{www.swisstopo.ch}}, dem Bundesamt für Landestopografie, in Wabern. Wir machen Karten. Unser Team macht Internetkarten - wie Google Maps\footnote{\href{https://maps.google.com}{maps.google.com}}, jedoch von der Schweiz für die Schweiz; und für alle anderen auch. Unsere Internetkarte, der Viewer, erfreut sich relativ grosser Beliebtheit
und beinhaltet ca. 800 Themen wie Wanderwege, Solarkataster und Luftfahrthindernisse. Lieber Leser\footnote{Im vorliegenden Dokument wird durchwegs der männliche Singular (Leser, Benutzer) als Ansprache verwendet. Diese Ansprache bezieht sich auf beide Geschlechter sowie gegebenenfalls mehrere Personen. Sie dient lediglich der leichteren Lesbarkeit der Semesterarbeit}, falls dir \href{https://map.geo.admin.ch}{map.geo.admin.ch} noch kein Begriff sein sollte, kann ich dir wärmstens empfehlen, darin zu schmökern. Es gibt viel zu entdecken und es ist gratis - ein Service Public.

\begin{figure}[H]
	\centering
	\href{https://s.geo.admin.ch/8a82499889}{
	\includegraphics[width=.80\textwidth]{hist_map_geo_admin}}
	\caption{Internetkarte des Bundes \emph{\href{https://s.geo.admin.ch/8a82499889}{map.geo.admin.ch}}. Hier ein Ausschnitt der Saane bei Kleinbösingen, das Luftaufnahmen von 1946 mit heute vergleicht.}
	\label{fig:map.geo.admin.ch}
\end{figure}


\subsection{GIS - Geographical Information Systems}
Wie erwähnt, arbeite ich als \emph{GIS-Spezialist}. Wobei mir der Titel \emph{Geo-Informatiker} besser gefällt: weil er die Begriffe \emph{Geografie} und \emph{Informatik} vereint. \emph{Geografie} kommt aus dem Griechischen und bedeutet Erdbeschreibung \autocite[14]{Schertenleib2004}. \emph{Informatik} ist die Wissenschaft von der systematischen Darstellung, Speicherung, Verarbeitung und Übertragung von Informationen \cite{Informatik2010}.

GIS ist ein Akronym für Geographical Information Systems. Es bedeutet im engsten Sinn eine Ansammlung von Computerprogrammen, die zur Bearbeitung von Karten und Geodaten verwendet werden. Geodaten sind nichts weiter als Daten mit einem räumlichen Bezug\footnote{mit Koordinaten (Nord/Ost, x/y)}. In einem weiteren Sinn deckt der Begriff GIS ein ganzes Fachgebiet ab, das sich mit Karten und Geodaten auskennt. Es ist also nicht nur ein Werkzeug, sondern ein Fachgebiet, das Kenntnisse über Datensammlung, Speicherung, Analyse und Darstellung innerhalb von vielen verschiedenen Themen mit einem räumlichen Bezug abdeckt. Typische Geodaten sind digitale Karten, Inventare und Register von Parzellen, Umweltfaktoren, Grenzen, Entwicklung, Planung etc., die einen räumlichen Bezug haben und dadurch in einem geografischen Zusammenhang analysiert und dargestellt werden können \autocite[15]{Balstroem}.

Es ist zwar weit von der klassischen Geografie zu Zeiten Alexanders von Humbold\footnote{Ein Forschungsreisender des 19 Jh. mit einem weit über Europa hinausreichenden Wirkungsfeld. Mehrjährige Forschungsreisen führten ihn nach Lateinamerika, USA und nach Zentralasien \cite{Kehlmann2005}} entfernt, aber es liegt auf der Hand, dass auch die Geografie Cloud Computing nutzt.