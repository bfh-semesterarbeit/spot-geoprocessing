\section{Ausgangslage}
Hier werden das Arbeitsumfeld und die Aufgaben beschrieben, aus denen sich die Problemstellung ergeben hat.

\subsection{swisstopo bei AWS}
Es liegt auf der Hand, dass die swisstopo als \emph{Geoinformationszentrum} auf Cloud Computing setzt. Die swisstopo nutzt Cloud Computing mit AWS\footnote{Amazon Web Services} seit mehr als 10 Jahren für den Betrieb des Geoportal des Bundes.
 
\textit{"Mit der BGDI\footnote{Bundesgeodateninfrastruktur: Viewer und andere Services} unter AWS können wir derzeit ca. eine Million Internetbenutzer pro Monat bedienen. Dank AWS können wir die zur Zuordnung neuer Server benötigte Zeit erheblich verkürzen und unseren Fokus auf echte Kundenanforderungen verstärken."} \cite{Christ2020}.


\subsection{Publikation von Geodaten}
Wie bereits erwähnt, können auf dem Viewer ca. 800 Themen wie Wanderwege, Solarkataster oder Luftfahrthindernisse angesehen werden. Unser Team publiziert diese Daten. Der Nachführungszyklus wie auch der Aufwand zur
Aufbereitung der Daten fürs Web sind unterschiedlich. Einige Daten werden manuell aufwändig aufbereitet, andere
stündlich automatisch nachgeführt.

\subsection{Web-Services}
Nebst der Publikation der Daten ist unser Team für den Betrieb und der Weiterentwicklung der Web-Services
und des Viewers verantwortlich. Der ganze Technologie Stack wurde schon länger nicht mehr grundlegend erneuert. Zurzeit wird
die gesamte Architektur analysiert und überarbeitet, um eine gestaffelte Migration auf eine neue
Lösung zu ermöglichen.
Einige Rahmenbedingungen dieser zukünftigen Architektur sind bereits klar: Das Geoportal des
Bundes wird weiterhin in der AWS Cloud betrieben werden, die Migration wird vor allem über
Microservices gestaffelt erfolgen, diese Services werden als Docker Container laufen, Amazon
Elsatic Kubernetes Service wird die Orchestrierung der Container übernehmen; und für Continuous
Integration wird AWS Codebuild/Pipeline zum Einsatz kommen.

\newpage

\subsection{Exkurs 3D Daten}
Die swisstopo erfasst und aktualisiert Daten mit einem räumlichen Bezug. Diese Geodaten sind die Basis für die Ableitung in eine Vielzahl von Produkten, wie die Landeskarten 1:25'000. Nebst Karten gibt es die Produktpalette der Landschaftsmodelle. Diese geben die Objekte der Landschaft im flexiblen Vektorformat wieder. Sie bestehen aus thematischen Ebenen (Bsp. Gebäude). Jede Ebene umfasst georeferenzierte Punkt-, Linien-, Flächen- oder 3D-Objekte. Jedes Objekt enthält Attribute und Beziehungen \cite{toposhop2010}.

Zu den Lanschaftsmodellen gehören Produkte wie swissTLM3D und swissBuildings3D. Im Viewer wird eine Auswahl von Themen aus eben diesen Landschaftsmodellen dargestellt: Zurzeit Gebäude, Bäume, Seilbahnen, Namen und das Terrain. Vor wenigen Jahren wurden diese 3D Daten mit einem grossen Effort medienwirksam publiziert.

\begin{figure}[H]
	\centering
	\href{https://s.geo.admin.ch/8a8ce63073}{
	\includegraphics[width=.80\textwidth]{bfh_3d}}
	\caption{Im Viewer werden zurzeit Gebäude, Bäume, Seilbahnen, Namen und das Terrain dargestellt. Um aktuell zu bleiben, müssen diese 3D Daten regelmässig nachgeführt werden.}
	\label{fig:bfh_3d}
\end{figure}

Seit dem Erfolg der Erstpublikation ist inzwischen etwas Zeit vergangen. Als die ersten Aktualisierungen der Daten anstanden, wurde den Beteiligten bewusst, dass sich diese nicht einfach so \emph{auf Knopfdruck} realisieren lässt: Seit der Erstpublikation hat es personelle Wechsel gegeben und punkto Dokumentation und Automatisierungsgrad wurden Lücken identifiziert.

Es gibt immer gute Gründe für \emph{technische Schulden}, wie in diesem Fall positive medienwirksame Reaktionen\footnote{Wie Bsp. auf watson.ch oder Twitter \cite{watson2018}}. Aber früher oder später müssen diese abgebaut werden, weil es einen direkten Einfluss auf die Wartbarkeit des Produktes hat \cite{technischeschulden2010}.

\section{Poblemstellung}

Es ist der swisstopo schon lange ein Anliegen, dass die Publikationsprozesse von Geodaten optimiert werden
sollen. Wann immer möglich, soll der Automatisierungsgrad erhöht werden.

Besonders aufwändig erweist sich zurzeit die Publikation von 3D Daten. Die manuelle Publikation
der 3D Daten benötigt eigene Tools, die auf einem performanten und somit teuren Rechner laufen
müssen. Ausserdem erfordert die Bereitstellung einen hohen Koordinationsaufwand zwischen der
Infrastruktur und der Entwicklung. Dabei passiert es, dass Mängel in den 3D Daten erst nach
beendeter Webpublikation bemerkt werden; und der ganze Publikationsprozess muss wieder von
vorne gestartet werden.
Auch dem Hersteller der 3D Daten (dem Bereich Topografie) wäre es ein Anliegen, wenn er diese
Daten selbst automatisch publizieren und prüfen könnte.

\subsection{Ist-Zustand der 3D Datenpublikation}
\subsubsection{Prozess der Publikation}
\begin{itemize}
\item Ausfindig machen des Zeitaufwandes pro Update bei uns und bei der IT
\item Topo an Kogis, Kogis an IT, Kogis an Topo ... 
\item Bild des Prozesses, der Infra
\item Manuelles Bereitstellen von EC2 Instanzen
\end{itemize}

\subsection{Aufwand der Prozessierung}


\subsection{Technologie Stack}


