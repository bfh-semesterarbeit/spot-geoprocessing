\section{Evaluation}
\subsection{Erfahrungen}
Die Anregungen aus dem Studium\footnote{CAS in Cloud Computing an der BFH BERN: Bsp. die Vorlesungen PaaS, Docker und Kubernetes} haben den Autor motiviert, die komplette Infrastruktur für den POC selber in einem eigenen AWS Testaccount aufzubauen. Da die Infrastruktur und das Processing deklarativ festgehalten wurden, kann ein Grossteil dieses Codes auch im Account des Betriebes ohne grosse Anpassungen wiederverwendet werden.

Im Betrieb hätte das Team und die IT Abteilung das nötige Know-How gehabt, um mit dieser Arbeit weiter zu kommen, als es jetzt der Fall ist. Aber der Autor hat sich bewusst dazu entschieden, die Arbeit möglichst selbständig zu realisieren und  diese Ressource zu schonen, indem er sie so wenig wie möglich genutzt hat. Es war für den Autor schon nur eine Hilfe, zu Wissen, dass die Möglichkeit einer Unterstützung da war. Dies führte dazu, dass der Autor Momente der Ratlosigkeit erleben musste\ref{har:perplexity}. Was nicht weiter schlimm war, da \textit{"Perplexity is the beginning of knowledge"}\autocite[33]{CloudNativ:1}.


\subsection{Kritische Punkte}
Bezüglich Authentifizierung hat der POC Technische Schulden.

\section{Zusammenfassung}


\section{Ausblick}