\section{Architektur}

\subsection{Analyse des Ist-Zustandes}
Vor allem beim Setup kann viel automatisiert (eingespart) werden. 
\begin{itemize}
\item Cronetab: Zippen und rsync
\item Cronetab: Run Instance
\end{itemize}

\subsection{Bewertungskriterien}
\begin{tabular}{ll}
    \textbf{Kosten einsparen} & Es darf sicher nicht teurer sein als die bisherige Lösung\\
    \textbf{Automatisierbar} & Die Lösung soll den Grad der Automatisierung möglichst weit vorantreiben\\
    \textbf{Einfach} & Jemand, der die Details der Prozessierung nicht kennt, soll diese mit einem minimalen
    Aufwand aufbauen und vorantreiben können.\\
    \textbf{Kein Exot} & Der Technologie Stack soll möglichst denjenigen der Microservices entsprechen\\
\end{tabular}


\subsubsection{Prozessing}
Da die Geodaten mittels Container prozessiert werden ... 

\subsubsection{Variante 1: Direkt auf Spot Instanz}
\begin{itemize}
\item Welches base AMI: Ubuntu - weil Wissen da ist, es direkt vom Internet her erreichbar ist...
\item Ansible oder AMI ? -> Ansible, weil damit weitere Schritte abgedeckt werden können.
\end{itemize}




\subsubsection{Variante 2: Kubernetes (EKS) auf Spot Instanzen erweitern}
Spot mit EKS. 

EBook auf meinem Reader: Job und Cronejob

\subsubsection{Variante 3: AWS Batch auf Spot Instanzen}

https://aws.amazon.com/de/blogs/compute/creating-a-simple-fetch-and-run-aws-batch-job/


\subsection{Bewertung}

\subsection{Architekturentscheid}
Rampup vor allem via Ansible und Cronejob
Ansible und Spot -> Wenn die Zeit noch reicht -> EKS -> Batch

TODO: Skizze
