\section{Architektur}

\subsection{Analyse des Ist-Zustandes}
\subsubsection{Bereitstellung der Rohdaten und Sicherstellung dessen Qualität}
Basierend auf der Aufwandeinschätzung des Ist-Zustandes des Kapitels \ref{aufwand_prozessierung} geht hervor, dass vor allem das Bereitstellen der Rohdaten und das Sicherstellen dessen Qualität aufwändig ist. 

\paragraph{Das Bereitstellen der Rohdaten} nimmt aufgrund der Datenmenge\footnote{Weil es sich um mehrere Millionen Dateien handelt} viel Zeit in Anspruch. Hier gäbe es folgende zwei Lösungsansätze:
\begin{enumerate}
\item Der Datenlieferant könnte direkt aufs EFS schreiben (Abb. \ref{fig:ist_zustand}, Nr. 5)
\item Automatischer Prozess via Cronjob oder oder Trigger, der die Kopier-Schritte (Abb. \ref{fig:ist_zustand}, Schritte Abb. \ref{fig:ist_zustand}) regelmässig bereitstellt.
\end{enumerate}

Wobei der Lösungsansatz simpler und somit weniger fehleranfällig zu sein scheint. Dieser Ansatz kann, sobald die swisstopo eine Hybrid Cloud hat, weiterverfolgt werden.

\paragraph{Sicherung der Qualität:} Weil es dem Lieferanten an Tools fehlt, um die bereitgestellten Daten inhaltlich zu Prüfen, werden Fehler 
häufig erst nach der Publikation entdeckt. Dies sicherlich auch, weil die Daten im Web an ein breites Publikum gelangen. Da keine inhaltliche Prüfung der Daten gemacht werden kann, könnte man sich überlegen, wie die Daten sonst noch geprüft werden könnten.

\subsubsection{Bereitstellung der Infrastruktur für die Prozessierung}
Die Rohdaten werden via Docker Image prozessiert. Jedes Mal wenn das Image fürs Prozessieren ausgeführt werden soll, muss die IT gebeten werden die nötige Infrastruktur bereitzustellen. Idealerweise findet sich eine Lösung, die diesen Schritt überflüssig macht.


\subsection{Bewertungskriterien}
\begin{tabular}{p{0.3\textwidth}p{0.7\textwidth}}
    \textbf{Kosten einsparen} & Änderungen des Prozesses sollen nicht teurer sein, als die bisherige Lösung\\
    \textbf{Automatisierbar} & Die Lösung soll den Grad der Automatisierung möglichst weit vorantreiben und dadurch so wenig Personalaufwand wie möglich beanspruchen.\\
    \textbf{Einfach} & Ein neuer Mitarbeiter soll die Lösung rasch verstehen und warten können.\\
    \textbf{Bestehende Technologie} & Obwohl der Anwendungsfall anders ist, soll der Technologie Stack möglichst demjenigen der Microservices entsprechen.\\
\end{tabular}

\subsection{Architekturentscheid}
Für den Architekturentscheid wurde auf eine Entscheidungsmatrix verzichtet. Dies Aufgrund der grossen Auswahl und der vielen Entscheidungen, die getroffen werden mussten. Die oben erwähnten Kriterien wurden in der Entscheidungsfindung mit einbezogen und weitere Argumente werden ausgeführt:

\subsubsection{Prozessierung auf Spot Instanzen}
Um Geodaten mit AWS Spot Instanzen zu prozessieren, hat der Autor drei Möglichkeiten identifiziert: \emph{Direkt auf der Spot Instanz}, \emph{Kubernetes (EKS)} und \emph{AWS Batch}. Gerne hätte der Autor alle drei Lösungen als POC weiterverfolgt, aber die Zeit dazu reichte leider nicht. Aus diesem Grund beschränkt sich der Autor lediglich auf eine Beschreibung der drei Möglichkeiten. Wobei die erste Möglichkeit als POC umgesetzt wurde.

\paragraph{Direkt auf einer Spot Instanz:}
Diese Möglichkeit bedingt nur geringe Anpassungen der bisherigen Prozessierung, weil die Daten im Prinzip wie bisher auf EC2 Instanz prozessiert werden; mit dem Nachteil, dass diese jederzeit durch eine andere Instanz ersetzt werden könnte.

Der Autor hat sich bei der Auswahl der Images\footnote{AMI: Amazon Machine Image.} für \emph{Ubuntu Server 18.04 LTS} entschieden. Dies weil das Prozessing bisher auf Ubuntu gelaufen ist und weil er keinen Grund sieht, etwas das funktioniert zu ändern. 

Bei der Provisionierung\footnote{Bereitstellung.} des Servers wurde der \emph{"Cloud-init"} Ansatz gewählt: Dieser Ansatz richtet das Ubuntu Image in der Startup Phase des Servers ein. Dies hat gegenüber einem eigenen AMI den Vorteil, dass die Spezialisierung\footnote{Das Einrichten.} in einer Textdatei festgehalten wird und somit jederzeit nachvollziehbar ist. Ausserdem muss bei einem Update des Basis Images kein neues Image gebaut werden.

\paragraph{Kubernetes (EKS) auf Spot Instanzen erweitern}
TODO: Weil die Microservices in Zukunft auf EKS laufen werden, wäre es sicher ein Möglichkeit, abzuklären, inwiefern sich ein Dataprozessing in dieser Umgebung einrichten lässt. In der Literatur hat der Autor folgende Elemente gefunden.

\begin{itemize}
\item Job
\item CronJob
\item Eigenes ReplicaSet?
\end{itemize}

EBook auf meinem Reader: Job und CronJob aber auch Kubernetes on AWS S. 88

\paragraph{AWS Batch auf Spot Instanzen}
Diese Möglichkeit wäre wohl die Naheliegendste, weil AWS Batch für genau diese Art von Aufgaben gebaut wurde. Im Internet hat der Autor einen schönes Beispiel gefunden, wie sich AWS Batch einsetzen lässt.
https://aws.amazon.com/de/blogs/compute/creating-a-simple-fetch-and-run-aws-batch-job/

