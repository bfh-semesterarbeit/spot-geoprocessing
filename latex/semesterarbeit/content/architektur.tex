\section{Architektur}

\subsection{Analyse des Ist-Zustandes}
\subsubsection{Bereitstellung der Rohdaten und Sicherstellung dessen Qualität}
Basierend auf der Aufwandeinschätzung des Ist-Zustandes des Kapitels \ref{aufwand_prozessierung} geht hervor, dass vor allem das Bereitstellen der Rohdaten und das Sicherstellen dessen Qualität aufwändig ist. 

\textbf{Das Bereitstellen der Rohdaten} nimmt aufgrund der Datenmenge\footnote{Weil es sich um mehrere Millionen Dateien handelt} viel Zeit in Anspruch. Hier gäbe es folgende zwei Lösungsansätze:
\begin{enumerate}
\item Der Datenlieferant könnte direkt aufs EFS schreiben (Abb. \ref{fig:ist_zustand}, Nr. 5)
\item Automatischer Prozess via Cronjob oder oder Trigger, der die Kopier-Schritte (Abb. \ref{fig:ist_zustand}, Schritte Abb. \ref{fig:ist_zustand}) regelmässig bereitstellt.
\end{enumerate}

\textbf{Sicherung der Qualität}
Weil es dem Lieferanten an Tools fehlt, um die bereitgestellten Daten inhaltlich zu Prüfen, werden Fehler 
häufig erst nach der Publikation entdeckt: Dies sicherlich auch, weil die Daten im Web an ein breites Publikum gelangen. Idealerweise

\subsubsection{Bereitstellung Infrastruktur fürs Prozessing}
Die Rohdaten werden via Docker-Image prozessiert. Damit die IT nicht jedes mal gebeten werden muss, Infrastruktur bereitzustellen, muss ein Weg gefunden der die Infrastrukur fürs Prozessing automatisch bereitstellt. 



Vor allem beim Setup kann viel automatisiert (eingespart) werden. 
\begin{itemize}
\item Cronetab: Zippen und rsync
\item Cronetab: Run Instance
\end{itemize}

\subsection{Bewertungskriterien}
\begin{tabular}{ll}
    \textbf{Kosten einsparen} & Es darf sicher nicht teurer sein als die bisherige Lösung\\
    \textbf{Automatisierbar} & Die Lösung soll den Grad der Automatisierung möglichst weit vorantreiben und so wenig Personal wie möglich beanspruchen.\\
    \textbf{Einfach} & Jemand, der die Details der Prozessierung nicht kennt, soll diese mit einem minimalen
    Aufwand aufbauen und vorantreiben können.\\
    \textbf{Kein Exot} & Der Technologie Stack soll möglichst denjenigen der Microservices entsprechen\\
\end{tabular}


\subsubsection{Prozessing}
Da die Geodaten mittels Container prozessiert werden ... 

\subsubsection{Variante 1: Direkt auf Spot Instanz}
\begin{itemize}
\item Welches base AMI: Ubuntu - weil Wissen da ist, es direkt vom Internet her erreichbar ist...
\item Ansible oder AMI ? -> Ansible, weil damit weitere Schritte abgedeckt werden können.
\end{itemize}




\subsubsection{Variante 2: Kubernetes (EKS) auf Spot Instanzen erweitern}
Spot mit EKS. 

EBook auf meinem Reader: Job und Cronejob

\subsubsection{Variante 3: AWS Batch auf Spot Instanzen}

https://aws.amazon.com/de/blogs/compute/creating-a-simple-fetch-and-run-aws-batch-job/


\subsection{Bewertung}

\subsection{Architekturentscheid}
Rampup vor allem via Ansible und Cronejob
Ansible und Spot -> Wenn die Zeit noch reicht -> EKS -> Batch

TODO: Skizze
