\section{Fachbegriffe und Abkürzungen}
\begin{table}[!htbp]
\begin{tabular}{p{0.2\textwidth}p{0.7\textwidth}}
    \textbf{Abkürzung} & \textbf{Definition}\\
    \textbf{Ansible} & \makecell[l]{Ein Open-Source Automatisierungs-Werkzeug zur Orchestrierung\\ und allgemeinen Konfiguration und Administration von Computern.}\\
    \textbf{AMI} & Amazon Machine Images: Images für virtuelle Server.\\
    \textbf{Availability Zone} & \makecell[l]{Jeder Amazon Rechenzentrumstandort (Region)\\
    besteht aus mehreren isolierten Zonen, den Availability Zones.}\\
    \textbf{AWS} & Amazon Web Services.\\
    \textbf{AWS CLI} & Das Command Line Interface, um AWS Ressourcen zu verwalten.\\
    \textbf{BGDI} & Bundesgeodateninfrastruktur.\\
    \textbf{Budget Instanzen} & Im Kontext dieser Arbeit ein Synonym für AWS Spot Instanzen.\\
    \textbf{EBS} & Block Storage: Speicher (für eine Instanz).\\
    \textbf{EC2} & Amazon Elastic Compute Cloud: Rechenkapazität, Speicher (und mehr).\\
    \textbf{EFS} & Amazon Elastic File System: Cloud NFS-Dateisystem.\\
    \textbf{Hybrid Cloud} & Eine Computerinfrastruktur, die Public Cloud und Private Cloud kombiniert nutzt.\\
    \textbf{IaaS} & Cloud Infrastructure as a Service: Infrastruktur \emph{"Pay as you go"} beziehen.\\
    \textbf{IAM} & Identity and Access Management: Verwaltung der Zugänge und der dazugehörenden Rechte.\\
    \makecell[l]{\textbf{On-Demand}\\ \textbf{Instanzen}} & \makecell[l]{Herkömmliche EC2 Instanzen. Der Begriff wird in dieser Arbeit manchmal\\ verwendet, um von EC2 Spot Instanzen unterscheiden zu können.}\\
	\textbf{PaaS} & Plattform as a Service.\\
	\textbf{POC} & Proof of Concept: Die Machbarkeit eines Produktes oder einer Idee aufzeigen.\\
	\makecell[l]{\textbf{S3}} & \makecell[l]{Amazon S3 (Simple Storage Service): Ein Filehosting-Dienst dessen Zugriff\\ über HTTP/HTTPS erfolgt.}\\
	\textbf{SaaS} & Software as a Service.\\
	\textbf{SSD} &  Solid-State-Disk - Halbleiterlaufwerk: Festplatte ohne bewegliche Teile, dafür mit kurzen Zugriffszeiten.\\
\end{tabular}
\caption{\label{tab:fachbegriffe}In der Arbeit verwendete Fachbegriffe und Abkürzungen}
\end{table}