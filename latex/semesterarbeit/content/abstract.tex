\section{Management Summary}
Wie aus dem Titel der Arbeit \emph{Geodatenverarbeitung mit Bugdet Instanzen} zu entnehmen ist, geht es bei dieser Semesterarbeit darum, Geodaten mittels Budget Instanzen zu verarbeiten: Geodaten sind Daten mit einem räumlichen Bezug, sie haben Koordinaten. Bugdet Instanzen sind günstige virtuelle Server vom Amazon AWS EC2 Angebot, sogenannte Spot Instanzen, die den Nachteil haben, dass sie innerhalb einer 2 minütigen Vorankündigung entfernt werden können.

Aus der Problemstellung geht hervor, dass die swisstopo die Publikationsvoränge von Geodaten fortlaufend optimieren möchte. Zurzeit erweist sich die Publikation von 3D Geodaten als besonders aufwändig, weil die manuelle Publikation spezielle Tools benötig, die auf performanten und somit teuren Rechnern laufen müssen. Ausserdem erfordert die Bereitstellungen einen hohen Koordinationsaufwand zwischen IT (Infrastruktur), Entwicklung und Auftraggeber.
Mit dieser Arbeit wird mittels Prototyp aufgezeigt, wie der Automatisierungsgrad erhöht werden könnte und wie bei der Verarbeitung Budget Instanzen zum Einsatz kommen könnten.

Um den Automatisierungsgrad wesentlich zu erhöhen, wurde die Datenverarbeitung mittels Automatisierungswerkzeug \emph{Ansible} in einem deklarativen Code beschrieben. Sobald eine Spot Instanz hochgefahren wird, wird das Ansible Skript automatisch gestartet.
Die einzelnen Verarbeitungsschritte werden in einer Textdatei auf einem separaten Laufwerk festgehalten. Falls eine Spot Instanz beendet werden sollte, kann die nächste automatisch gestartete Instanz die Verarbeitung beim letzten festgehaltenen Schritt fortführen.

Mit dem Prototypen konnte gezeigt werden, dass sich Spot Instanzen gut eignen, um Geodaten zu verarbeiten. Durch die Automatisierung würden Personalstunden wegfallen und die Fehleranfälligkeit kleiner werden.
Das relative Sparpotential beim Einsatz von Bugdet Instanzen ist enorm, beim gewählten Use Case 76\%. In absoluten Zahlen machen die Einsparungen einer Datenpublikation gerade einmal 15 \$ aus. Jedoch mit der Zeit würde sich dieser absolute Betrag mit zunehmendem Einsatz von Spot Instanzen ganz sicher aufsummieren.


\pagebreak