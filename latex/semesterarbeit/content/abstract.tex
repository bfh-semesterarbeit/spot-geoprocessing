\section{Management Summary}
Wie der Titel \emph{Geodatenverarbeitung mit Budget Instanzen} andeutet, werden in dieser Semesterarbeit Geodaten mittels Budget Instanzen verarbeitet: Geodaten sind Daten mit einem räumlichen Bezug, sie haben Koordinaten. Budget Instanzen sind virtuelle Server vom Amazon AWS EC2 Angebot, sogenannte Spot Instanzen, die zwar günstig sind, aber den Nachteil haben, dass sie innerhalb einer 2-minütigen Vorankündigung entfernt werden können.

Aus der Problemstellung geht hervor, dass die swisstopo die Publikationsvorgänge von Geodaten fortlaufend optimieren möchte. Zurzeit erweist sich die Publikation von 3D Geodaten als besonders aufwändig, weil die manuelle Publikation spezielle Tools benötigt, die auf performanten und somit teuren Rechnern laufen müssen. Ausserdem erfordert die Bereitstellungen einen hohen Koordinationsaufwand zwischen IT (Infrastruktur), Entwicklung und Auftraggeber.
Mit dieser Arbeit wird mittels eines Prototyps aufgezeigt, wie der Automatisierungsgrad erhöht werden kann und wie bei der Verarbeitung Budget Instanzen zum Einsatz kommen könnten.

Um den Automatisierungsgrad wesentlich zu erhöhen, wurden die Datenverarbeitungsschritte mittels Automatisierungswerkzeug Ansible in einem deklarativen Code beschrieben.\\ Die Spot Instanzen können über eine Flottenanfrage gesteuert werden. Die Flottenanfrage wurde so konfiguriert, dass sie dafür sorgt, dass immer eine Spot Instanz mit vordefinierten Mindestanforderungen zur Verfügung gestellt wird. Sobald eine Spot Instanz hochgefahren wurde, wird die Datenverarbeitung automatisch gestartet.
Die einzelnen Verarbeitungsschritte werden auf einem separaten Laufwerk festgehalten. Falls eine Spot Instanz beendet werden sollte, kann die nächste von der Flottenanfrage automatisch gestartete Instanz die Verarbeitung beim letzten festgehaltenen Schritt fortführen.

Mit dem Prototypen konnte gezeigt werden, dass sich Spot Instanzen gut eignen, um Geodaten zu verarbeiten. Durch die Automatisierung würden Personalstunden wegfallen und die Fehleranfälligkeit kleiner werden. Beim Use Case wurden alle erfassten 3D Gebäude der Schweiz auf Spot Instanzen innerhalb von 18 Stunden verarbeitet.
So konnte verdeutlicht werden, dass das relative Sparpotential beim Einsatz von Budget Instanzen gegenüber gewöhnlichen (On-Demand) Instanzen enorm ist, beim gewählten Use Case 76\%. Jedoch in absoluten Zahlen macht die Einsparung gerade einmal 15\$ aus. Dennoch sind Spot Instanzen aus wirtschaftlicher Sicht ein interessantes Angebot und ein vermehrter Einsatz würde sich lohnen.

\pagebreak