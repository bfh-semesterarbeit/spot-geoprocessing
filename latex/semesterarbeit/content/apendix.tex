\section{Kanban}\label{appendix:kanban}
\begin{figure}[H]
	\centering
	\includegraphics[width=.95\textwidth]{kanban}
	\caption{Klassisches Kanban auf \emph{github.com}.}
	\label{fig:Klassisches Kanban}
\end{figure}

\section{Projektplan}\label{appendix:projektplan}
\begin{figure}[H]
	\centering
	\href{https://docs.google.com/spreadsheets/d/1zKTZgt4BW736G0xRfU9o3vWYwAJj-8nzFvGsPR7yJ_0/edit?usp=sharing}{
	\includegraphics[width=.95\textwidth]{projektplanung}}
	\caption{\href{https://docs.google.com/spreadsheets/d/1zKTZgt4BW736G0xRfU9o3vWYwAJj-8nzFvGsPR7yJ_0/edit?usp=sharing}{Projektplan. Die orangen Meilensteine wurden von der BFH vorgegeben.}}
	\label{fig:Projektplan}
\end{figure}

\section{Konfigurationen und Kommandos}
\subsection{EFS auf EC2-Instanz mounten}
Anhand einer Anleitung, einem sogenannten Walktrough, wurde via AWS CLI\footnote{AWS Command Line Interface: Ein kommandozeilenorientiertes Werkzeug.} ein
EFS an eine EC2-Instanz gemountet und die Rohdaten wurden schon einmal dorthin kopiert. In der Verarbeitung bildet dieses EFS den Ausgang der Datenverarbeitung.
\appendCode{bash}{EFS auf EC2-Instanz mounten.}{src/walktrough_ec2_and_efs.sh}


\subsection{Von der SPOT Instanz aus abfragen, was ihr Status ist}\label{appendix:restful}
Von der Instanz aus können via RESTful API Metadaten der Instanz abgefragt werden. Bezüglich Interrupt einer Spot Instanz kann der Zustand \emph{none}, \emph{hibernate}, \emph{stop} oder \emph{terminate} sein. \emph{none}, wenn nichts ansteht. Von da an, wo klar ist, dass die Instanz abgestellt werden wird, kann der Zeitpunkt ausgelesen werden.
\appendCode{bash}{Status der Instanz abfragen.}{src/rest_instance_metadata.sh}

\subsection{Well-formed XML}
Einfaches Skript zum testen, ob alle Dateien well-formed XML sind.
\appendCode{python}{XML testen.}{src/up-and-running-dataprocessing/ansible/tests/test_xml_wellformed.py}

\subsection{Geodatenverarbeitunsschritte in Ansible}
Der ganze Code zum hier aufgelisteten Ausschnitt kann auf \href{https://github.com/bfh-semesterarbeit/spot-geoprocessing}{github.com} eingesehen, resp. verwendet, werden:
\appendCode{make}{Nebst dem Setup werden die vier hier Publikationsschritte von Ansible gesteuert.}{src/up-and-running-dataprocessing/ansible/playbooks/processing.yml}


%\section{Für die Semesterarbeit verwendete Software}
%\begin{itemize}
%\item JabRef: Verwaltung des Literaturverzeichnisses (BibTeX).
%\item Gummi: LaTeX Editor.
%\item AWS CLI: Für das Bereitstellen der AWS Infrastruktur.
%\item jq: Für das Filtern von JSON (vor allem von AWS CLI Antworten).
%\item git: Versionsmanagement der Textdateien.
%\item Google Spreadsheet: Für den \nameref{chap:projektplan}.
%\end{itemize}
